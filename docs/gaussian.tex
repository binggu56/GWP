\documentclass[11pt]{article}
\usepackage{graphicx}
\usepackage{amssymb}
\usepackage{epstopdf}
\usepackage{mystyle}
\usepackage{bm}
\usepackage{amsmath}
\DeclareGraphicsRule{.tif}{png}{.png}{`convert #1 `dirname #1`/`basename #1 .tif`.png}

\title{Trajectory guided gaussian basis method}
\author{Bing Gu}
\date{}                                           % Activate to display a given date or no date

\begin{document}
\maketitle
\section{Gaussian Basis}
One possible way to solve time-dependent \se is to expand the initial wavepacket as a linear combination of gaussian basis. 
\be g_{m}(x,t) = \sqrt[4] {\frac{a}{\pi}} \exp\left (-\frac{\alpha}{2}(x-q_{m}(t))^{2}+\im p_{m}(t)(x-q_{m}(t) \right) \ee 
\be |\psi_{0} \ket= \sum_{n=1,...,N}c_{n}(t=0)|g_{n}(q_{n},p_{n},t=0) \ket  \ee
\be \bra g_{m} | \psi_{0} \ket = \sum_{n}  c_{n}\bra g_{m} | g_{n} \ket, ~~~ m=1,...,N. \ee
where $N$ is the number of basis used to project the initial wavepacket. \\ 


Initial coefficients $c_{n}(0)$ can be obtained by solving the matrix equation 
\be \bf{M}\bm{c} =  \bm{b} \ee  
where 
\be {\bf M}_{mn} = \bra g_{m} | g_{n} \ket, ~~~{\bf b} = \{ \bra g_{1}  | \psi_{0} \ket, \dots , \bra g_{n} | \psi_{0} \ket \}. \ee 
%\subsection{}
Normalization of the wavepacket is conversed in the propagation.
\be N = \bra \psi | \psi \ket = \sum_{mn} \bm c_{m}^{*}{\bf M}_{mn}\bm c_{n}, \ee 
\be \frac{dN}{dt} = 0 \ee 
%It can be proved by using the Hermitian property of hamiltonian ${\bf H}$. 
%(\bm{c^{*}})^{T}\bm({\phi}^{*})^{T}\bm{\phi} \bm c = (\bm c^{*})^{T} \bf{M} \bm c \ee
Define 
\be \bm c = \{ c_{1}, c_{2},\dots, c_{N} \} \ee 
and 
\be \bm{ \phi} = \{ g_{1}(q_{1},p_{1}),\dots, g_{N}(q_{N},p_{N}) \}, \ee 
Wavefunction at time $t$ can be written as 
\be \psi(x,t) = \bm c^{T}(t)\bm \phi(t) \label{eq:expand} \ee  
if we substitute Eq. (\ref{eq:expand}) into time-dependent \se, propagation of the initial wavepacket can be transformed into the evolution of the coefficients $c_{n}(t)$ and the motion of $(p_{n}(t),q_{n}(t))$ of gaussian wave packets. 
The equations for evolution of $\bm c$ will be 
\be \im {\bf M} \dot{c} = ({\bf H} - \im \dot{\bf M})\bm c , \label{eq:dcdt} \ee 
where 
\be \dot{\bf M}_{mn} = \bra g_{m} | \dot{g_{n}} \ket \ee 
and ${\bf H}$ is the hamiltonian matrix,   
\be   {\bf H}_{mn} = \bra g_{m} | -\frac{\hbar^{2}}{2m}\grad^{2} + V(x) | g_{n} \ket . \ee
Potential energy is expanded into second-order Taylor series at  position of each trajectory
\be V(x) = V(q_{n}) + \grad V(q_{n} ) (x-q_{n}) + \frac{\grad^{2} V(q_{n})}{2}(x-q_{n})^{2} + O((x-q_{n})^{3}) \label{eq:taylor} \ee 
Substituting Eq. (\ref{eq:taylor}) to Eq. (\ref{eq:dcdt}) and ignore the third-order term, we will obtain 
\be {\bf H}_{mn} = \bra g_{m} | d_{0}+d_{1}(x-q_n)+d_{2}(x-q_n)^2 | g_{n} \ket . \ee
\be d_{0} = V(x_{n}) - \frac{p^{2}_{n}-\alpha}{2m},~~d_{1} = -\grad U(x_{n}), ~~ d_{2} = \frac{1}{2} \left (\grad^{2}V(x_{n}) - \frac{\alpha^{2}}{m} \right) \ee 
     
\end{document} 

